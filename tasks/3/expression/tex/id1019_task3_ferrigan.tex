\documentclass[a4paper,11pt]{article}

\usepackage[utf8]{inputenc}
\usepackage[english]{babel}

\usepackage{hyperref}
\usepackage{amsmath}

% Code highligting
% \usepackage{minted}
\usepackage[outputdir=output/tex]{minted} % iom min makefile

\newenvironment{longlisting}{\captionsetup{type=listing}}{}
% \renewcommand\listoflistingscaption{Källkod....}
\renewcommand\listoflistingscaption{List of source codes}
%\setmintedinline[sql]{breaklines=true,breakanywhere=true} % necessary for breakanywhere to work later on.

\usepackage{graphicx}
\usepackage{pgf}
\usepackage{wrapfig}
\usepackage[font=footnotesize,labelfont=bf,skip=2pt]{caption}
\usepackage{subcaption}

\usepackage{pgfplots}
\pgfplotsset{compat=1.18}

% Spacing
\usepackage{titlesec}
\titlespacing*{\section}{0pt}{2ex plus 1ex minus .2ex}{1ex plus .2ex}


\begin{document}

\title{
    Evaluating an expression
\\\small{Programmering II, ID1019, VT24 P1}
}
\author{Vincent Ferrigan \href{mailto:ferrigan@kth.se}{ferrigan@kth.se}}

% \date{\today}
\date{Spring Term 2023}

\maketitle

\section*{Introduction}
\label{sec:introduction}

\section*{Methods}
\label{sec:methods}

\subsection*{Literature Study}
\label{subsec:literature}
\subsubsection*{Tools and packages}

\label{subsec:tools}
\subsubsection*{The overall Work-flow}

\label{subsec:workflow}
\section*{Result}
\label{sec:result}

\section*{Discussion}
\label{sec:discussion}

\end{document}

% \begin{longlisting}
%   \inputminted[
%       label=Q1-Function,
%       linenos=true,
%       bgcolor=lightgray,
%       firstline=28,
%       lastline=45,
% %        frame=single,
%       fontsize=\footnotesize,
%   ]{sql}{../../src/db/analytics/analytics.sql}
%   \caption{
%     Function for filtering and ordering output-table from view on given year.
%     }
%   \label{listing:q1_func}
% \end{longlisting}

% \begin{minted}{elixir}
%   def append([], b) do b end
%   def append([h|t], b) do
%     [h | append(t, b)]
%   end  
% \end{minted}

% If you want to include a program statement in running text you can do this
% using for example teletype-text: {\tt append([1,2,3],[4,5])}.
% using for example mintinline: \mintinline{elixir}{append([1,2,3],[3])}

% \begin{table}[h]
% \begin{center}
% \begin{tabular}{l|c|c}
% \textbf{prgm} & \textbf{run time} & \textbf{ratio}\\
% \hline
%   dummy      &  115 &     1.0\\
%   union      &  535 &     4.6\\
%   tailr      &  420 &     3.6\\
% \end{tabular}
% \caption{Union and friends, list of 50000 elements, run time in micro seconds}
% \label{tab:table1}
% \end{center}
% \end{table}


% \section*{Graphs}

% Once you start to generate graphs make sure that they are readable and
% have sensible information on the axes.

% There are many ways to generate graphs but you want to use a way that
% minimize manual work. My tool over the years has been {\em Gnuplot}
% and if you do not have a favorite tool you could give it a try. The
% Gnuplot program is a stand alone program that will generate a
% graph that you then can include in you report.

% If you work with Gnuplot you should write the commands needed to
% generate a diagram in a small script. Take a look in the file {\tt
%   fib.p} and you will see how the diagram in Fig.\ref{fig:images} was
% created from the data given in {\tt fib.dat} (the {\tt .png} file was
% generated from {\tt fib.pdf} using the Linux {\tt convert}
% program found in {\tt imagemagick}).

% When you include graphs you should make sure that the images you
% include are not raster images (gif, png etc) but a vector image that
% scales when you zoom-in. In Fig.\ref{fig:images} you see the same
% graph saved as a raster image (png) compared to a vector graphic
% image. You might not see the difference but if you zoom-in you will
% see that the vector image scales.

% \begin{figure}[h]
%   \centering
%   \begin{subfigure}{.5\textwidth}
%     \centering
%     \includegraphics[scale=0.45]{fib.png}
%     \caption{using raster graphics}
%   \end{subfigure}%
%   \begin{subfigure}{.5\textwidth}
%     \centering
%     \includegraphics[scale=0.45]{fib.pdf}
%     \caption{using vector graphics.}
%   \end{subfigure}
%   \caption{Difference in image formats.}
%   \label{fig:images}
% \end{figure}

% An alternative to including a graph produced by a separate program is
% to describe the graph in \LaTeX. This can be done using the TikZ
% library. This library is used to create all types of graphics and the
% learning curve is quite steep. The benefit is that the \LaTeX document
% becomes self contained and that you are in complete control over the result.

% The data can either be written in the latex source file but better read
% from a separate file. Reading from a separate file makes it easier to
% combine the output from a benchmark with the report. If you construct
% your benchmark to produce a file with the x and y values in columns
% you can plot them using a simple {\tt \textbackslash addplot}
% command. If you do changes to your program you simply run the
% benchmark again and re-compile the \LaTeX file.

% \begin{figure}
%   \centering
%   \begin{tikzpicture}
%     \begin{axis}[
%       xmin=12, xmax=28, ymin=0, ymax=3500,
%       xlabel=n, ylabel={time in $\mu s$},
%       width=8cm, height=6cm]
%       \addlegendentry{run time fib(n)};
%       \addplot[] table {fib.dat};
%     \end{axis}

%   \end{tikzpicture}
%   \caption{The same graph using TikZ}
%   \label{fig:tikz}
% \end{figure}

% The graph in Fig.\ref{fig:tikz} is generated using Tikz and as you can
% see, I know have the time in "$\mu s$" instead of in "us".

