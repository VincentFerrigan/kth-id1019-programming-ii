\documentclass[a4paper,11pt]{article}

\usepackage[utf8]{inputenc}
\usepackage[english]{babel}

\usepackage{amsmath}

% Code highligting
% \usepackage{minted}
\usepackage[outputdir=output/tex]{minted} % iom min makefile

\newenvironment{longlisting}{\captionsetup{type=listing}}{}
% \renewcommand\listoflistingscaption{Källkod....}
\renewcommand\listoflistingscaption{List of source codes}
%\setmintedinline[sql]{breaklines=true,breakanywhere=true} % necessary for breakanywhere to work later on.

\usepackage{graphicx}
\usepackage{pgf}
\usepackage{wrapfig}
\usepackage[font=footnotesize,labelfont=bf,skip=1pt]{caption}
\usepackage{subcaption}

\usepackage{pgfplots}
\pgfplotsset{compat=1.18}

\usepackage{pgfplotstable}
\usepackage{booktabs}

% Spacing
\usepackage{titlesec}
%\titlespacing*{\section}{0pt}{2ex plus 1ex minus .2ex}{1ex plus .2ex}
%\titlespacing*{\subsection}{0pt}{1ex plus 1ex minus .2ex}{1ex plus .2ex}

\usepackage{hyperref}

\begin{document}
\title{
    Hot Springs
    \\Part Ons
\\\small{Programmering II, ID1019, VT24 P1}
}
\author{Vincent Ferrigan \href{mailto:ferrigan@kth.se}{ferrigan@kth.se}}

\date{\today}

\maketitle

\section*{Introduction}
\label{sec:introduction}
The chief objective of this two-part assignment is to explore the significance
and utility of \emph{Dynamic Programming}.

Part one, covered in this report, consists of solving a puzzle with
\emph{''brute force''}.
The task is to solve the problem of analyzing a field of
hot springs with varying conditions.
The objective is to parse descriptions of rows of springs,
where each spring can be operational (denoted by \verb|.|),
damaged (denoted by \verb|#|), or of unknown status (denoted by \verb|?|),
and then to determine the number of valid arrangements of springs
based on a sequence of numbers that indicate the consecutive counts of damaged springs.

The second part will add memoization to the mix i.e.
solving the puzzle with dynamic programming.
It will also include benchmarks, comparing the brute force solution to the dynamic one.

%The second part, covered by this report, add memoization i.e.
%solving the puzzle with dynamic programming.
%It will also include benchmarks, comparing the brute force solution to the dynamic one.

This assignment is based on the instruction
\href{https://people.kth.se/~johanmon/courses/id1019/seminars/springs/springs.pdf}{'Hot springs'}
by course examiner Johan Montelius.
The Mix-project for this assignment, including all relative functions, Unit-test and benchmarks can be found on GitHub:
\href{https://github.com/VincentFerrigan/kth-id1019-programming-ii/tree/main/tasks/5/day12}{Repo Programming II - Day12}% TODO FUNKAR DEN? ÄNDRA NAMN

\section*{Methods}
\label{sec:methods}
\subsection*{Literature Study}
\label{subsec:literaturestudy}
The pre-recorded lectures on
\href{https://canvas.kth.se/courses/44911/assignments/syllabus}{''v7 Komplexitet och dynamisk programmering''},
given by the course examiner, were reviewed.
Elixir-syntax and similar topics were acquired
from both the
\href{https://elixir-lang.org/docs.html}{Elixir official documentation}
and the free Elixir Tutorial
\href{https://elixirschool.com/en}{Elixir School
website}.
The author also relied on textbooks found on
\href{https://learning.oreilly.com}{O'Reilly Media},
especially Brute Tate's
\href{https://learning.oreilly.com/library/view/programmer-passport-elixir/9781680509649/}{Programmer Passport: Elixir}.
Chapter 14 \emph{Dynamic Programming} in the Advance Alorithms book
\emph{Programming Problems} by Bradley Green was also studied.

\subsubsection*{Tools and packages}
\label{subsec:tools}
All code was written in \emph{IntelliJ IDEA}.
Quick-fixes and editing was, however, done in \emph{Vim}.
\emph{GIT} and \emph{GitHub} were used for version control.
Tests were performed with Elixir's built-in test framework \emph{ExUnit.}
The report was written in \LaTeX.
For exploration and benchmarking, the interactive environment and notebook
\href{https://livebook.dev/}{\emph{Livebook}}
was heavily used.

\subsubsection*{Test Driven Development}
\label{subsec:ttd}
The development was an iterative approach with ''trail and errors''.
The author practiced
\href{https://www.elixirwiki.com/wiki/Test-Driven_Development_in_Elixir}{Test Driven Development}
(TDD).
The basic steps follow the \emph{Red-Green-Refactor cycle};
writing failing tests (Red), make them pass (Green),
and finally refactor the code.
For example, before creating the three higher-order functions,
one can first write a couple of test cases, as illustrated in the test-example below.
%TODO.
\inputminted[
    label=SAMPLE TESTS,
    firstline=5,
    lastline=13,
    xleftmargin=-3mm,  % Adjust this value as needed
    fontsize=\footnotesize,
]{elixir}{../test/day12_test.exs}
\noindent The
\mintinline{elixir}{run_sample/1},
runs the sample input and calculates the total number of valid spring arrangements.
It takes an input string representing the sample spring descriptions,
separates each line into spring conditions and damaged springs sequence,
then calculates the total number of valid arrangements using a brute force approach.
\inputminted[
label=map,
firstline=53,
lastline=60,
xleftmargin=-3mm,  % Adjust this value as needed
fontsize=\footnotesize,
]{elixir}{../lib/day12.ex}
\inputminted[
    label=map,
    firstline=50,
    lastline=51,
    xleftmargin=-3mm,  % Adjust this value as needed
    fontsize=\footnotesize,
]{elixir}{../lib/day12.ex}

\section*{Result}
\label{sec:result}
All
\href{https://github.com/VincentFerrigan/kth-id1019-programming-ii/tree/main/tasks/5/day12/lib/day12.ex}{code (lib/day12.ex)} and % TODO FUNKAR DEN? ÄNDRA NAMN
\href{https://github.com/VincentFerrigan/kth-id1019-programming-ii/tree/main/tasks/5/day12/test/day12_test.exs}{tests (test/day12\_test.exs)}
can be found on GitHub.
.....
For example, given the sample
\mintinline{elixir}{"????.######..#####. 1,6,5"}, the
\mintinline{elixir}{run_sample/1} mentioned in the
~\href{sec:methods}{method section} above, would do the following:

\begin{enumerate}
    \item The input is split and parsed by the function
\mintinline{elixir}{parse_line/1} into a pattern (list of chars)
\mintinline[fontsize=\small]{elixir}{~c"????.######..#####"} and a sequence
\mintinline{elixir}{[1, 6, 5]}.
\inputminted[
    label=map,
    firstline=76,
    lastline=82,
    xleftmargin=-3mm,  % Adjust this value as needed
    fontsize=\footnotesize,
]{elixir}{../lib/day12.ex}
\inputminted[
    label=map,
    firstline=73,
    lastline=74,
    xleftmargin=-6mm,  % Adjust this value as needed
    fontsize=\footnotesize,
]{elixir}{../lib/day12.ex}

\item Initiate Solving: The
\mintinline{elixir}{brute_force_solve/1} function calls
\mintinline{elixir}{count/4} with the parsed pattern, sequence, false for needing a dot,
and 0 for the initial count of needed hashes.

\item Recursive Counting: The
\mintinline{elixir}{count/4} function begins processing the pattern:

For each \verb|?|, it considers two paths: as a \verb|.| and as a \verb|#|.
This branching occurs for the first four \verb|?| characters,
creating a combination of possibilities.
When encountering the sequence of six \verb|#|,
it matches this with the sequence part \verb|[6]|.
Since there's an exact match, this path continues.
The dots (\verb|..|) are passed without issue,
as they don't conflict with any requirements.
The next sequence of five \verb|#| matches the last part of the sequence \verb|[5]|.

\item Calculate Results: Each valid path through the pattern that matches the sequence
increments the count.
In cases where a \verb|?| can be either a \verb|.| or a \verb|#|
without violating the sequence constraints, both possibilities are explored,
potentially doubling the count for each \verb|?|.
However, any sequence that doesn't match exactly results in a branch of calculation
returning 0, pruning that path from the total count.

\item Final Count: The sum of all valid paths gives the total count of arrangements.
The algorithm efficiently prunes impossible configurations early through its
recursive checks.
\end{enumerate}

\noindent This step-by-step explanation illustrates a simplified view of the algorithm's logic. The real power of Elixir's pattern matching and recursion shines in handling complex branching with concise code.
%TODO, Go through an example

%\inputminted[
%label=map,
%firstline=56,
%lastline=57,
%xleftmargin=-3mm,  % Adjust this value as needed
%fontsize=\footnotesize,
%]{elixir}{../lib/day12.ex}

%\mintinline{elixir}{}

%\inputminted[
%label=map,
%firstline=174,
%lastline=176,
%xleftmargin=-3mm,  % Adjust this value as needed
%fontsize=\footnotesize,
%]{elixir}{../lib/day12.ex}
\section*{Discussion}
\label{sec:discussion}

\subsection*{Charlists}
Please note that charlists are created using the
$\sim$c Sigil.
Elixir \emph{strings} are enclosed with double quotes,
while \emph{charlists} are enclosed with single quotes.
Each value in a charlist, according to
\href{https://elixirschool.com/en/lessons/basics/strings#charlists-1}{Elixir School},
is the \emph{Unicode code point} of a character whereas in a binary,
the codepoints are encoded as \emph{UTF-8}.
The first 128 characters of UTF-8 have the same binary values as \emph{ASCII},
making ASCII text valid UTF-8.
However, according to
\href{https://hexdocs.pm/elixir/binaries-strings-and-charlists.html}{Elixirs Official Documentation},
the list is only printed as a sigil if all code points are within the ASCII range.
    %TODO, write about brute force being a solution to solve a puzzle with a small sample
    % especially when solving AOC
\end{document}