\documentclass[a4paper,11pt]{article}

\usepackage[utf8]{inputenc}
\usepackage[english]{babel}

\usepackage{amsmath}

% Code highligting
% \usepackage{minted}
\usepackage[outputdir=output/tex]{minted} % iom min makefile

\newenvironment{longlisting}{\captionsetup{type=listing}}{}
% \renewcommand\listoflistingscaption{Källkod....}
\renewcommand\listoflistingscaption{List of source codes}
%\setmintedinline[sql]{breaklines=true,breakanywhere=true} % necessary for breakanywhere to work later on.

\usepackage{graphicx}
\usepackage{pgf}
\usepackage{wrapfig}
\usepackage[font=footnotesize,labelfont=bf,skip=1pt]{caption}
\usepackage{subcaption}

\usepackage{pgfplots}
\pgfplotsset{compat=1.18}

\usepackage{pgfplotstable}
\usepackage{booktabs}

% Spacing
\usepackage{titlesec}
%\titlespacing*{\section}{0pt}{2ex plus 1ex minus .2ex}{1ex plus .2ex}
%\titlespacing*{\subsection}{0pt}{1ex plus 1ex minus .2ex}{1ex plus .2ex}

\usepackage{hyperref}

\begin{document}

\title{
    Reduce and friends
    \\Part Two
\\\small{Programmering II, ID1019, VT24 P1}
}
\author{Vincent Ferrigan \href{mailto:ferrigan@kth.se}{ferrigan@kth.se}}

 \date{\today}

\maketitle

\section*{Introduction}
\label{sec:introduction}
The chief objective of this two-part assignment is to explore the significance
and utility of \emph{Dynamic Programming}.

Part one, covered in this report, consists of solving a puzzle with
\emph{''brute force''}.
The task is to solve the problem of analyzing a field of
hot springs with varying conditions.
The objective is to parse descriptions of rows of springs,
where each spring can be operational (denoted by \verb|.|),
damaged (denoted by \verb|#|), or of unknown status (denoted by \verb|?|),
and then to determine the number of valid arrangements of springs
based on a sequence of numbers that indicate the consecutive counts of damaged springs.

The second part will add memoization to the mix i.e.
solving the puzzle with dynamic programming.
It will also include benchmarks, comparing the brute force solution to the dynamic one.

%The second part, covered by this report, add memoization i.e.
%solving the puzzle with dynamic programming.
%It will also include benchmarks, comparing the brute force solution to the dynamic one.

This assignment is based on the instruction
\href{https://people.kth.se/~johanmon/courses/id1019/seminars/springs/springs.pdf}{'Hot springs'}
by course examiner Johan Montelius.
The Mix-project for this assignment, including all relative functions, Unit-test and benchmarks can be found on GitHub:
\href{https://github.com/VincentFerrigan/kth-id1019-programming-ii/tree/main/tasks/5/day12}{Repo Programming II - Day12}% TODO FUNKAR DEN? ÄNDRA NAMN

\section*{Methods}
\label{sec:methods}
\subsection*{Literature Study}
\label{subsec:literaturestudy}
The pre-recorded lectures on
\href{https://canvas.kth.se/courses/44911/assignments/syllabus}{''v7 Komplexitet och dynamisk programmering''},
given by the course examiner, were reviewed.
Elixir-syntax and similar topics were acquired
from both the
\href{https://elixir-lang.org/docs.html}{Elixir official documentation}
and the free Elixir Tutorial
\href{https://elixirschool.com/en}{Elixir School
website}.
The author also relied on textbooks found on
\href{https://learning.oreilly.com}{O'Reilly Media},
especially Brute Tate's
\href{https://learning.oreilly.com/library/view/programmer-passport-elixir/9781680509649/}{Programmer Passport: Elixir}.
Chapter 14 \emph{Dynamic Programming} in the Advance Alorithms book
\emph{Programming Problems} by Bradley Green was also studied.

\subsubsection*{Tools and packages}
\label{subsec:tools}
All code was written in \emph{IntelliJ IDEA}.
Quick-fixes and editing was, however, done in \emph{Vim}.
\emph{GIT} and \emph{GitHub} were used for version control.
Tests were performed with Elixir's built-in test framework \emph{ExUnit.}
The report was written in \LaTeX.
For exploration and benchmarking, the interactive environment and notebook
\href{https://livebook.dev/}{\emph{Livebook}}
was heavily used.

\subsubsection*{Test Driven Development}
\label{subsec:ttd}
The development was an iterative approach with ''trail and errors''.
The author practiced
\href{https://www.elixirwiki.com/wiki/Test-Driven_Development_in_Elixir}{Test Driven Development}
(TDD).
The basic steps follow the \emph{Red-Green-Refactor cycle};
writing failing tests (Red), make them pass (Green),
and finally refactor the code.
For example, before creating the three higher-order functions,
one can first write a couple of test cases, as illustrated in the test-examples below.

%TODO.
%The product of an empty list should be one (the empty product),
%since one is the identity element for multiplication.
%The sum of an empty list must also result in its identity element, which is zero.
\inputminted[
    label=SAMPLE TESTS,
    firstline=5,
    lastline=13,
    xleftmargin=-3mm,  % Adjust this value as needed
    fontsize=\footnotesize,
]{elixir}{../test/day12_test.exs}

\section*{Result}
\label{sec:result}
All
\href{https://github.com/VincentFerrigan/kth-id1019-programming-ii/tree/main/tasks/5/day12/lib/day12.ex}{code (lib/day12.ex)} and % TODO FUNKAR DEN? ÄNDRA NAMN
\href{https://github.com/VincentFerrigan/kth-id1019-programming-ii/tree/main/tasks/5/day12/test/day12_test.exs}{tests (test/day12\_test.exs)}
can be found on GitHub.
.....
%TODO, Go through an example

%\mintinline{elixir}{}
\inputminted[
label=map,
firstline=174,
lastline=176,
xleftmargin=-3mm,  % Adjust this value as needed
fontsize=\footnotesize,
]{elixir}{../lib/day12.ex}
\section*{Discussion}
\label{sec:discussion}
\end{document}