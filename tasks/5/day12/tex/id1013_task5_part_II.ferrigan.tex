\documentclass[a4paper,11pt]{article}

\usepackage[utf8]{inputenc}
\usepackage[english]{babel}

\usepackage{amsmath}

% Code highligting
% \usepackage{minted}
\usepackage[outputdir=output/tex]{minted} % iom min makefile

\newenvironment{longlisting}{\captionsetup{type=listing}}{}
% \renewcommand\listoflistingscaption{Källkod....}
\renewcommand\listoflistingscaption{List of source codes}
%\setmintedinline[sql]{breaklines=true,breakanywhere=true} % necessary for breakanywhere to work later on.

\usepackage{graphicx}
\usepackage{pgf}
\usepackage{wrapfig}
\usepackage[font=footnotesize,labelfont=bf,skip=1pt]{caption}
\usepackage{subcaption}

\usepackage{pgfplots}
\pgfplotsset{compat=1.18}

\usepackage{pgfplotstable}
\usepackage{booktabs}

% Spacing
\usepackage{titlesec}
%\titlespacing*{\section}{0pt}{2ex plus 1ex minus .2ex}{1ex plus .2ex}
%\titlespacing*{\subsection}{0pt}{1ex plus 1ex minus .2ex}{1ex plus .2ex}

\usepackage{hyperref}

\begin{document}
\title{
    Hot Springs
    \\Part Two
\\\small{Programmering II, ID1019, VT24 P1}
}
\author{Vincent Ferrigan \href{mailto:ferrigan@kth.se}{ferrigan@kth.se}}

\date{\today}

\maketitle

\section*{Introduction}
\label{sec:introduction}
The chief objective of this two-part assignment is to explore the significance
and utility of \emph{Dynamic Programming}.

The first part consisted of solving a puzzle with
\emph{''brute-force''}.
The task was to solve the problem of analyzing a field of
hot springs with varying conditions.
The objective was to parse descriptions of rows of springs,
where each spring can be operational (denoted by \verb|'.'|),
damaged (denoted by \verb|'#'|), or of unknown status (denoted by \verb|'?'|),
and then to determine the number of valid arrangements of springs
based on a sequence of numbers that indicate the consecutive counts of damaged springs.

The second part, covered by this report, is to utilize dynamic programming
by adding memoization.

For this puzzle, dynamic programming can significantly reduce the computational
complexity by memorizing (or caching) the results of subproblems,
thereby avoiding redundant calculations and speeding up the overall solution process.

It will also include benchmarks, comparing the brute-force solution to the dynamic one.
Extending the input with a multiplier will be the technique used to explore the problem space.

This assignment is based on the instruction
\href{https://people.kth.se/~johanmon/courses/id1019/seminars/springs/springs.pdf}{'Hot springs'}
by course examiner Johan Montelius.
The Mix-project for this assignment, including all relative functions, Unit-test and benchmarks can be found on GitHub:
\href{https://github.com/VincentFerrigan/kth-id1019-programming-ii/tree/main/tasks/5/day12}{Repo Programming II - Day12}% TODO FUNKAR DEN? ÄNDRA NAMN

\section*{Methods}\label{sec:methods}
\subsection*{Literature Study}
\label{subsec:literaturestudy}
The pre-recorded lectures on
\href{https://canvas.kth.se/courses/44911/assignments/syllabus}{''v7 Komplexitet och dynamisk programmering''},
given by the course examiner, were reviewed.
Elixir-syntax and similar topics were acquired
from both the
\href{https://elixir-lang.org/docs.html}{Elixir official documentation}
and the free Elixir Tutorial
\href{https://elixirschool.com/en}{Elixir School
website}.
The author also relied on textbooks found on
\href{https://learning.oreilly.com}{O'Reilly Media},
especially Brute Tate's
\href{https://learning.oreilly.com/library/view/programmer-passport-elixir/9781680509649/}{Programmer Passport: Elixir}.
Chapter 14 \emph{Dynamic Programming} in the Advance Alorithms book
\emph{Programming Problems} by Bradley Green was also studied.

\subsubsection*{Tools and packages}
\label{subsec:tools}
All code was written in \emph{IntelliJ IDEA}.
Quick-fixes and editing was, however, done in \emph{Vim}.
\emph{GIT} and \emph{GitHub} were used for version control.
Tests were performed with Elixir's built-in test framework \emph{ExUnit.}
For the visualization of benchmarking results, \emph{GNUPlot} was used.
The report was written in \LaTeX.

For exploration and benchmarking, the interactive environment and notebook
\href{https://livebook.dev/}{\emph{Livebook}}
was heavily used.

\subsubsection*{Test Driven Development}\label{sec:ttd}
The development was an iterative approach with ''trail and errors''.
The author practiced
\href{https://www.elixirwiki.com/wiki/Test-Driven_Development_in_Elixir}{Test Driven Development}
(TDD).
The basic steps follow the \emph{Red-Green-Refactor cycle};
writing failing tests (Red), make them pass (Green),
and finally refactor the code.
For example, before creating the three higher-order functions,
one can first write a couple of test cases, as illustrated in the test-example below.
\inputminted[
    label=SAMPLE TESTS,
    firstline=15,
    lastline=23,
    xleftmargin=-3mm,  % Adjust this value as needed
    fontsize=\footnotesize,
]{elixir}{../test/day12_test.exs}
\noindent The
\mintinline{elixir}{run_dynamic_sample/2},
runs the sample input and calculates the total number of valid spring arrangements.

It takes an input string representing the sample spring descriptions,
extends each line's pattern and damaged springs sequence according to the provided multiplier,
then calculates the total number of valid arrangements using a dynamic programming approach.
\inputminted[
label=run_dynamic_sample/2,
firstline=209,
lastline=217,
xleftmargin=-3mm,  % Adjust this value as needed
fontsize=\footnotesize,
]{elixir}{../lib/day12.ex}
\inputminted[
    label=iex run_dynamic_sample/2,
    firstline=206,
    lastline=207,
    xleftmargin=-3mm,  % Adjust this value as needed
    fontsize=\footnotesize,
]{elixir}{../lib/day12.ex}

\section*{Result}
\label{sec:result}
In this section, a step-by-step explanation will be given to illustrate
a simplified view of the brute-force solution's algorithm's logic.
All
\href{https://github.com/VincentFerrigan/kth-id1019-programming-ii/tree/main/tasks/5/day12/lib/day12.ex}{code (lib/day12.ex)}, % TODO FUNKAR DEN? ÄNDRA NAMN
\href{https://github.com/VincentFerrigan/kth-id1019-programming-ii/tree/main/tasks/5/day12/test/day12_test.exs}{tests (test/day12\_test.exs)}
\href{https://github.com/VincentFerrigan/kth-id1019-programming-ii/tree/main/tasks/5/day12/data}{graphs (.png)}, and % TODO FUNKAR DEN? ÄNDRA NAMN
\href{https://github.com/VincentFerrigan/kth-id1019-programming-ii/tree/main/tasks/5/day12/data}{tables (.dat)}% TODO FUNKAR DEN? ÄNDRA NAMN
can be found on GitHub.

Given the sample
\mintinline{elixir}{"????.######..#####. 1,6,5"} and the multiplier $5$, the
\mintinline{elixir}{run_dynamic_sample/2} function mentioned in the
~\hyperref[sec:ttd]{method section} above, would do the following:
\begin{enumerate}
    \item The job of
\mintinline{elixir}{run_dynamic_sample/2} is to prepare the input for processing.
It does this by splitting the input string into lines
(in this case, there's only one line),
item extending each line according to the multiplier,
item parsing each line into a pattern and a sequence of damaged springs,
item and then calculating the total number of valid arrangements using a dynamic programming approach.
    \item Given the input string
\mintinline{elixir}{`"????.######..#####. 1,6,5"`} and the multiplier
\mintinline{elixir}{`5`},
the function
\mintinline{elixir}{extend_input/2}
will repeat the pattern and the sequence five times.
To ensure the pattern and sequence are repeated correctly,
the pattern is extended by concatenating it with itself five times
(with an extra \verb|'?'| for each join to maintain separation),
and the sequence is similarly repeated.
    \inputminted[
        label=extend_input,
        firstline=183,
        lastline=189,
        xleftmargin=-3mm,  % Adjust this value as needed
        fontsize=\footnotesize,
    ]{elixir}{../lib/day12.ex}
    \inputminted[
        label=iex extend_input,
        firstline=180,
        lastline=181,
        xleftmargin=-3mm,  % Adjust this value as needed
        fontsize=\footnotesize,
    ]{elixir}{../lib/day12.ex}
    \item After the input string is extended,
\mintinline{elixir}{parse_line/1} parses the extended string into a pattern
(a list of characters representing operational, damaged,
or unknown spring conditions) and a sequence (a list of integers representing the
counts of consecutive damaged springs).
    \inputminted[
        label=parse_line/1,
        firstline=86,
        lastline=93,
        xleftmargin=-3mm,  % Adjust this value as needed
        fontsize=\footnotesize,
    ]{elixir}{../lib/day12.ex}
    \inputminted[
        label=iex parse_line/1,
        firstline=83,
        lastline=84,
        xleftmargin=-3mm,  % Adjust this value as needed
        fontsize=\footnotesize,
    ]{elixir}{../lib/day12.ex}
    \item the function
\mintinline{elixir}{dynamic/1} kicks off the dynamic programming approach.
It receives the parsed line (pattern and sequence) and starts the process
of calculating the total number of valid configurations that match the given
sequence of damaged springs.
It uses memoization to avoid recalculating the number of valid arrangements
for the same pattern and sequence state.
    \inputminted[
        label=dynamic/1,
        firstline=248,
        lastline=261,
        xleftmargin=-3mm,  % Adjust this value as needed
        fontsize=\footnotesize,
    ]{elixir}{../lib/day12.ex}
    \inputminted[
        label=iex dynamic/1,
        firstline=245,
        lastline=246,
        xleftmargin=-3mm,  % Adjust this value as needed
        fontsize=\footnotesize,
    ]{elixir}{../lib/day12.ex}
    \item The heavy lifting is done by
\mintinline{elixir}{count/5}.
it is called recursively with different subsets of the pattern and sequence,
along with a memoization table
(\mintinline{elixir}{`mem`}) to store already computed results.
It handles different cases:
    \begin{itemize}
        \item End of pattern and sequence with no remaining damaged springs needed: count as $1$ valid arrangement.
        \item End of pattern but still expecting damaged springs or vice versa: count as $0$ (invalid arrangement).
        \item Encountering a damaged spring (\verb|`#`|) or operational spring (\verb|`.`|) and updating the state accordingly.
        \item Handling an unknown spring (\verb|`?`|) by exploring both possibilities (damaged or operational)
        and summing the valid arrangements for each choice.
   \end{itemize}
    \inputminted[
        label=count/5,
        firstline=263,
        lastline=296,
        xleftmargin=-3mm,  % Adjust this value as needed
        fontsize=\footnotesize,
    ]{elixir}{../lib/day12.ex}
    \item \textbf{Memoization:} Throughout the recursive calls,
results for specific states
(\mintinline{elixir}{pattern, sequence, needs_dot, need_hashes}) are stored in
\mintinline{elixir}{`mem`}.
If a state is encountered again, the stored result is used instead of recalculating.
\end{enumerate}

\noindent For our specific input
\mintinline{elixir}{"????.######..#####. 1,6,5"} and multiplier $5$,
the steps would be as follows:
\begin{itemize}
    \item Extend the input according to the multiplier \verb|`5`|.
This would make the pattern and sequence five times longer than the original.
    \item Parse the extended input to separate the pattern from the sequence.
    \item Apply the dynamic programming approach to calculate the total number of valid combinations,
leveraging memoization to optimize the process.
\end{itemize}

%\inputminted[
%label=map,
%firstline=56,
%lastline=57,
%xleftmargin=-3mm,  % Adjust this value as needed
%fontsize=\footnotesize,
%]{elixir}{../lib/day12.ex}

%\mintinline{elixir}{}

%\inputminted[
%label=map,
%firstline=174,
%lastline=176,
%xleftmargin=-3mm,  % Adjust this value as needed
%fontsize=\footnotesize,
%]{elixir}{../lib/day12.ex}
\section*{Discussion}\label{sec:discussion}
\subsection*{Charlists}
Please note that charlists are created using the
$\sim$c Sigil.
Elixir \emph{strings} are enclosed with double quotes,
while \emph{charlists} are enclosed with single quotes.
Each value in a charlist, according to
\href{https://elixirschool.com/en/lessons/basics/strings#charlists-1}{Elixir School},
is the \emph{Unicode code point} of a character whereas in a binary,
the codepoints are encoded as \emph{UTF-8}.
The first $128$ characters of UTF-8 have the same binary values as \emph{ASCII},
making ASCII text valid UTF-8.
However, according to
\href{https://hexdocs.pm/elixir/binaries-strings-and-charlists.html}{Elixirs Official Documentation},
the list is only printed as a sigil if all code points are within the ASCII range.
%TODO, write about brute force being a solution to solve a puzzle with a small sample
% especially when solving AOC
\subsection*{Brute force vs Dynamic Programming}
As mentioned in the previous report covering part I,
\emph{the brute-force approach} was a good strategy for solving the first part of the assignment,
and I guess, the first part of any Advent Of Code.
It involved exploring every possible combination of
spring states (operational, damaged, or unknown) to find all valid
arrangements that match the given sequence of damaged springs.
The time complexity of this approach might have been exponential $(O(2^n))$
for $n$ unknown
springs, making it highly inefficient for larger inputs.
As the input
length or the complexity of the sequence increases, the number of
combinations to explore grows exponentially, leading to significant
performance issues.

\emph{Dynamic Programming (DP)} breaks this problem down into smaller,
manageable subproblems, solving each once and storing their results for
future use (\emph{memoization}).
This approach significantly reduces the number of
calculations by avoiding the repetition of work done for overlapping
subproblems.
It can offer polynomial time complexity, which is much
more efficient than the exponential time complexity of brute-force methods,
especially for large inputs.

The problem of calculating valid spring arrangements exhibits the property of
overlapping subproblems, where the same subproblem appears multiple times
during the computation.
As mentioned, a brute force approach would redundantly solve these
subproblems each time they occur.
The dynamic programming solves each unique subproblem
once and uses
memoization to store these results.
When the same subproblem occurs again, it
retrieves the stored result instead of recalculating it.
This drastically
reduces the computational effort and time.

\subsubsection*{summa summarum}
For small input sizes, a brute force approach might be feasible and
straightforward to implement.
However, as the input size or the complexity of
the sequence of damaged springs increases, the brute force approach quickly
becomes impractical due to:
\begin{itemize}
    \item Exponential Growth in Computation Time. With each additional unknown
spring, the number of possible configurations doubles, leading to an
exponential increase in the number of scenarios to check.
    \item Memory Constraints. Storing and processing a massive number of
configurations can also lead to memory constraints.
    \item Inefficiency. Most of the computational effort in a brute-force approach
is wasted on recalculating outcomes for combinations that have already been
explored.
    \end{itemize}
In summary, dynamic programming is preferred in this case due to its efficiency
in handling complex inputs by avoiding redundant calculations, thereby making
the solution scalable and practical for larger problem instances where a brute
force approach would not work.

Solving part II of day 12 with brute force would have taken an
unsufferable amount of time.
\end{document}