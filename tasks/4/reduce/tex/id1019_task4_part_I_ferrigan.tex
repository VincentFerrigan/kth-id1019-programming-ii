\documentclass[a4paper,11pt]{article}

\usepackage[utf8]{inputenc}
\usepackage[english]{babel}

\usepackage{amsmath}

% Code highligting
% \usepackage{minted}
\usepackage[outputdir=output/tex]{minted} % iom min makefile

\newenvironment{longlisting}{\captionsetup{type=listing}}{}
% \renewcommand\listoflistingscaption{Källkod....}
\renewcommand\listoflistingscaption{List of source codes}
%\setmintedinline[sql]{breaklines=true,breakanywhere=true} % necessary for breakanywhere to work later on.

\usepackage{graphicx}
\usepackage{pgf}
\usepackage{wrapfig}
\usepackage[font=footnotesize,labelfont=bf,skip=1pt]{caption}
\usepackage{subcaption}

\usepackage{pgfplots}
\pgfplotsset{compat=1.18}

\usepackage{pgfplotstable}
\usepackage{booktabs}

% Spacing
\usepackage{titlesec}
%\titlespacing*{\section}{0pt}{2ex plus 1ex minus .2ex}{1ex plus .2ex}
%\titlespacing*{\subsection}{0pt}{1ex plus 1ex minus .2ex}{1ex plus .2ex}

\usepackage{hyperref}

\begin{document}

\title{
    Reduce and friends
\\\small{Programmering II, ID1019, VT24 P1}
}
\author{Vincent Ferrigan \href{mailto:ferrigan@kth.se}{ferrigan@kth.se}}

% \date{\today}
\date{Spring Term 2023}

\maketitle

\section*{Introduction}
\label{sec:introduction}
The chief objective of the assignment is to explore the significance and utility
of higher-order functions.

Part one consists of tackling list manipulations through
''traditional methods'' to discover/identify and discuss underlying patterns.
Here, both regular and tail recursive solutions will be demonstrated and compared.

The second part re-implements these operations by using higher-order functions
i.e. implementing generic solutions.
This will help the student to understand and apply the three main patterns of
higher-order functions: map, reduce, and filter.

The final task involves the use of the pipe operator \mintinline{elixir}{|>}
to streamline
function calls and improve code readability.
The pipe operator is the main operator for function composition in Elixir.

This assignment is based on the instruction
\href{https://people.kth.se/~johanmon/courses/id1019/seminars/reduce/reduce.pdf}{'Reduce and friends'}
by course examiner Johan Montelius.
The Mix-project for this assignment, including all relative functions, Unit-test and benchmarks can be found on GitHub:
\href{https://github.com/VincentFerrigan/kth-id1019-programming-ii/tree/main/tasks/4/reduce}{Repo Programming II - Reduce}% TODO FUNKAR DEN? ÄNDRA NAMN

\section*{Methods}
\label{sec:methods}
\subsection*{Literature Study}
\label{subsec:literaturestudy}
The pre-recorded lectures on
\href{https://canvas.kth.se/courses/44911/assignments/syllabus}{''v6 Högre ordningens funktioner''},
given by the course examiner, were reviewed.
Elixir-syntax and similar topics were acquired
from both the
\href{https://elixir-lang.org/docs.html}{Elixir official documentation}
and the free Elixir Tutorial
\href{https://elixirschool.com/en}{Elixir School
website}.
The author also relied on textbooks found on
\href{https://learning.oreilly.com}{O'Reilly Media},
especially Brute Tate's
\href{https://learning.oreilly.com/library/view/programmer-passport-elixir/9781680509649/}{Programmer Passport: Elixir}.

\subsubsection*{Tools and packages}
\label{subsec:tools}
All code was written in \emph{IntelliJ IDEA}.
Quick-fixes and editing was, however, done in \emph{Vim}.
\emph{GIT} and \emph{GitHub} were used for version control.
Tests were performed with Elixir's built-in test framework \emph{ExUnit.}
The report was written in \LaTeX.

\subsubsection*{The overall Work-flow}
\label{subsec:workflow}
The development was an iterative approach with ''trail and errors''.
The author practiced
\href{https://www.elixirwiki.com/wiki/Test-Driven_Development_in_Elixir}{Test Driven Development}
(TDD).
The basic steps follow the \emph{Red-Green-Refactor cycle};
writing failing tests (Red), make them pass (Green),
and finally refactor the code.
....
%TODO HÄR ÄR DU  NU::::::::
%TODO HÄR ÄR DU  NU::::::::
%TODO HÄR ÄR DU  NU::::::::
%TODO HÄR ÄR DU  NU::::::::

% %TODO exempel på ett test
%For example, before creating a helper-function that reduces rational numbers,
%one can first write a test-case.
%As illustrated in the test-example below, the result is expected to be
%$\frac{3}{2}$ rather than $\frac{6}{4}$.
%\begin{minted}[fontsize=\footnotesize]{elixir}
%test "Evaluate 3x/4, where x = 2" do
%  e = {:div, {:mul, {:num, 3}, {:var, :x}}, {:num, 4}}
%  env = %{x: 2}
%  assert Expression.eval(e, env) == {:q, 3, 2}
%end
%\end{minted}

% TODO SKRIV OM HUR DU DELADE UPP KODEN ETC....??
\section*{Result}
\label{sec:result}
Since all
\href{https://github.com/VincentFerrigan/kth-id1019-programming-ii/tree/main/tasks/4/reduce/lib/reduce.ex}{code (lib/reduce.ex)} and % TODO FUNKAR DEN? ÄNDRA NAMN
\href{https://github.com/VincentFerrigan/kth-id1019-programming-ii/tree/main/tasks/4/reduce/test/reduce_test.exs}{tests (test/reduce\_test.exs)}
could be found on GitHub, the author has chosen only to % TODO.........

%The head function
%\mintinline{elixir}{}
%......
%\begin{minted}[fontsize=\footnotesize]{elixir}
%iex>
%iex>
%
%
%:ok
%\end{minted}
%This function .....

\subsection*{Part ONE}
\label{subsec:res-part1}

\subsubsection*{Part Two}
\label{subsec:res-part2}

\section*{Discussion}
\label{sec:discussion}

\subsection*{Part ONE}
\label{subsec:dis-part1}

\subsubsection*{Part Two}
\label{subsec:dis-part2}
\end{document}

% regular recursion involves functions that make their recursive call but still
% have some computation to do after the call returns, potentially leading to
% larger memory usage due to the call stack. Tail recursion, on the other hand,
% is a specific form of recursion where the recursive call is the last operation
% in the function. This allows for optimizations such as tail call optimization
% (TCO), where the compiler can reuse the current function's stack frame for the
% recursive call, significantly reducing memory usage and preventing stack
% overflow errors. Tail recursion is therefore considered more efficient and
% performant for certain algorithms, especially in languages and environments
% that support TCO.
