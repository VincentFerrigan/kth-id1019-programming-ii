\documentclass[a4paper,11pt]{article}

\usepackage[utf8]{inputenc}
\usepackage[english]{babel}

\usepackage{amsmath}

% Code highligting
% \usepackage{minted}
\usepackage[outputdir=output/tex]{minted} % iom min makefile

\newenvironment{longlisting}{\captionsetup{type=listing}}{}
% \renewcommand\listoflistingscaption{Källkod....}
\renewcommand\listoflistingscaption{List of source codes}
%\setmintedinline[sql]{breaklines=true,breakanywhere=true} % necessary for breakanywhere to work later on.

\usepackage{graphicx}
\usepackage{pgf}
\usepackage{wrapfig}
\usepackage[font=footnotesize,labelfont=bf,skip=1pt]{caption}
\usepackage{subcaption}

% Table
\usepackage{booktabs}
\usepackage{longtable}
\usepackage{makecell} % Include makecell package

\renewcommand\theadalign{bc} % Center align the column headers
\renewcommand\theadfont{\bfseries} % Make column headers bold
\renewcommand\theadgape{\Gape[4pt]} % Add some spacing around the headers
\setcellgapes{3pt} % Spacing for the cell content

% For item space etc.
\usepackage{enumitem}

\usepackage{pgfplots}
\pgfplotsset{compat=1.18}

\usepackage{pgfplotstable}
\usepackage{booktabs}

% Spacing
\usepackage{titlesec}
%\titlespacing*{\section}{0pt}{2ex plus 1ex minus .2ex}{1ex plus .2ex}
%\titlespacing*{\subsection}{0pt}{1ex plus 1ex minus .2ex}{1ex plus .2ex}

\usepackage{hyperref}

\begin{document}
\title{
    Morse Encoding
\\\small{Programmering II, ID1019, VT24 P1}
}
\author{Vincent Ferrigan \href{mailto:ferrigan@kth.se}{ferrigan@kth.se}}

\date{\today}
\maketitle

\section*{Introduction}
\label{sec:introduction}
The Mix-project for this assignment, including all relative functions, Unit-test and benchmarks can be found on GitHub:
\href{https://github.com/VincentFerrigan/kth-id1019-programming-ii/tree/main/tasks/9/morse}{Repo Programming II - Morse}% TODO FUNKAR DEN? ÄNDRA NAMN

\section*{Methods}\label{sec:methods}
\subsubsection*{Tools and packages}
\label{subsec:tools}
All code was written in \emph{IntelliJ IDEA}.
Quick-fixes and editing was, however, done in \emph{Vim}.
\emph{GIT} and \emph{GitHub} were used for version control.
Tests were performed with Elixir's built-in test framework \emph{ExUnit.}
The report was written in \LaTeX.

\subsubsection*{Test Driven Development}\label{sec:ttd}
The development was an iterative approach with ''trail and errors''.
The author practiced
\href{https://www.elixirwiki.com/wiki/Test-Driven_Development_in_Elixir}{Test Driven Development}
(TDD).
The basic steps follow the \emph{Red-Green-Refactor cycle};
writing failing tests (Red), make them pass (Green),
and finally refactor the code.
% For example, before creating ranges for part II,
% one can first write a couple of test cases, as illustrated in the test-example below.
%TODO.
\inputminted[
    label=Tests hello,
    firstline=5,
    lastline=17,
    xleftmargin=-3mm,  % Adjust this value as needed
    fontsize=\footnotesize,
]{elixir}{../test/morse_test.exs}
\noindent
\inputminted[
    label=encode tests,
    firstline=19,
    lastline=30,
    xleftmargin=-3mm,  % Adjust this value as needed
    fontsize=\footnotesize,
]{elixir}{../test/morse_test.exs}

\section*{Result}
\subsection*{Decoding Example}
\label{sec:result}
   \inputminted[
       label=iex decode/2,
       firstline=19,
       lastline=21,
       xleftmargin=-3mm,  % Adjust this value as needed
       fontsize=\footnotesize,
   ]{elixir}{../lib/morse.ex}
\noindent 
\subsection*{Encoding Example}
   \inputminted[
       label=iex encode/1,
       firstline=61,
       lastline=62,
       xleftmargin=-3mm,  % Adjust this value as needed
       fontsize=\footnotesize,
   ]{elixir}{../lib/morse.ex}
\noindent 
All
\href{https://github.com/VincentFerrigan/kth-id1019-programming-ii/tree/main/tasks/9/morse/lib/morse.ex}{code (lib/morse.ex)} and
\href{https://github.com/VincentFerrigan/kth-id1019-programming-ii/tree/main/tasks/9/morse/test/morse_test.exs}{tests (test/morse\_test.exs)}, 
can be found on the author's
\href{https://github.com/VincentFerrigan/kth-id1019-programming-ii/}{Github)}.


\end{document}