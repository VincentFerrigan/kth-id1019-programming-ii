\documentclass[a4paper,11pt]{article}

\usepackage[utf8]{inputenc}
\usepackage[english]{babel}

\usepackage{amsmath}

% Code highligting
% \usepackage{minted}
\usepackage[outputdir=output/tex]{minted} % iom min makefile

\newenvironment{longlisting}{\captionsetup{type=listing}}{}
% \renewcommand\listoflistingscaption{Källkod....}
\renewcommand\listoflistingscaption{List of source codes}
%\setmintedinline[sql]{breaklines=true,breakanywhere=true} % necessary for breakanywhere to work later on.

\usepackage{graphicx}
\usepackage{pgf}
\usepackage{wrapfig}
\usepackage[font=footnotesize,labelfont=bf,skip=1pt]{caption}
\usepackage{subcaption}

\usepackage{pgfplots}
\pgfplotsset{compat=1.18}

\usepackage{pgfplotstable}
\usepackage{booktabs}

% Spacing
\usepackage{titlesec}
%\titlespacing*{\section}{0pt}{2ex plus 1ex minus .2ex}{1ex plus .2ex}
%\titlespacing*{\subsection}{0pt}{1ex plus 1ex minus .2ex}{1ex plus .2ex}

\usepackage{hyperref}

\begin{document}
\title{
    Huffman
\\\small{Programmering II, ID1019, VT24 P1}
}
\author{Vincent Ferrigan \href{mailto:ferrigan@kth.se}{ferrigan@kth.se}}

\date{23 february 2024}
\maketitle

\section*{Introduction}
\label{sec:introduction}
%TODO

The Mix-project for this assignment, including all relative functions, Unit-test and benchmarks can be found on GitHub:
\href{https://github.com/VincentFerrigan/kth-id1019-programming-ii/tree/main/tasks/8/huffman}{Repo Programming II - Huffman}% TODO FUNKAR DEN? ÄNDRA NAMN

\section*{Methods}\label{sec:methods}
\subsection*{Literature Study}
\label{subsec:literaturestudy}
Elixir-syntax and similar topics were acquired
from both the
\href{https://elixir-lang.org/docs.html}{Elixir official documentation}
and the free Elixir Tutorial
\href{https://elixirschool.com/en}{Elixir School
website}.
The author also relied on textbooks found on
\href{https://learning.oreilly.com}{O'Reilly Media},
especially Brute Tate's
\href{https://learning.oreilly.com/library/view/programmer-passport-elixir/9781680509649/}{Programmer Passport: Elixir}.

\subsubsection*{Tools and packages}
\label{subsec:tools}
All code was written in \emph{IntelliJ IDEA}.
Quick-fixes and editing was, however, done in \emph{Vim}.
\emph{GIT} and \emph{GitHub} were used for version control.
Tests were performed with Elixir's built-in test framework \emph{ExUnit.}
The report was written in \LaTeX.
For exploration and benchmarking, the interactive environment and notebook
\href{https://livebook.dev/}{\emph{Livebook}}
was heavily used.

\subsubsection*{Test Driven Development}\label{sec:ttd}
The development was an iterative approach with ''trail and errors''.
The author practiced
\href{https://www.elixirwiki.com/wiki/Test-Driven_Development_in_Elixir}{Test Driven Development}
(TDD).
The basic steps follow the \emph{Red-Green-Refactor cycle};
writing failing tests (Red), make them pass (Green),
and finally refactor the code.
% For example, before creating ranges for part II,
% one can first write a couple of test cases, as illustrated in the test-example below.
%TODO.
\inputminted[
    label=??????
    firstline=1,
    lastline=1,
    xleftmargin=-3mm,  % Adjust this value as needed
    fontsize=\footnotesize,
]{elixir}{../test/huffman_test.exs}
One can also set up tests for the sample data, given by Advent of Code, prior to coding.
\inputminted[
    label=??????
    firstline=1,
    lastline=1,
    xleftmargin=-3mm,  % Adjust this value as needed
    fontsize=\footnotesize,
]{elixir}{../test/huffman_test.exs}
%\noindent The
%\mintinline{elixir}{run_sample/1},

\section*{Result}
\label{sec:result}
Since all
\href{https://github.com/VincentFerrigan/kth-id1019-programming-ii/tree/main/tasks/8/day05/lib/huffman.ex}{code (lib/day12.ex)} and % TODO FUNKAR DEN? ÄNDRA NAMN
\href{https://github.com/VincentFerrigan/kth-id1019-programming-ii/tree/main/tasks/8/huffman/test/huffman_test.exs}{tests (test/day12\_test.exs)}
can be found on the author's
\href{https://github.com/VincentFerrigan/kth-id1019-programming-ii/}{Github)},
this section will only give a brief overview for each part.
%TODO write what
%a simplified view of the brute-force solution's algorithm's logic.

%    \inputminted[
%        label=brute_force_solve/2,
%        firstline=111,
%        lastline=114,
%        xleftmargin=-3mm,  % Adjust this value as needed
%        fontsize=\footnotesize,
%    ]{elixir}{../lib/day05.ex}
%    \inputminted[
%        label=iex brute_force_solve/2,
%        firstline=108,
%        lastline=109,
%        xleftmargin=-3mm,  % Adjust this value as needed
%        fontsize=\footnotesize,
%    ]{elixir}{../lib/day05.ex}

\section*{Discussion}\label{sec:discussion}
\end{document}